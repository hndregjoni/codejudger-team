\section{Executive summary}

\subsection{Project summary}
\textbf{CodeJudger} is a platform that facilitates coding competitoins, and coding problem evaluation. 
A collection of problems (curated by the users of the system) offered, that the users of the system
can attempt to solve. Solving a problem will come with additional benefits (a scoreboard, achievements, etc.).
The problems can also have additional attributes to them, which allows for a more flexible organization of the problems
(i.e. courses, timed events, etc.).

Users are allowed to solve a problem in a language of their choice, and the range of choices is large, and is
easily extendable to allow for the addition of more languages. The integration of additional languages and/or tools
is seamless, and doesn't allow for unintended behaviour that could lead to potential vulnerabilities.
\subsection{Purpose and scope of this specification}
The purpose of this specification is twofold:
\begin{enumerate}
    \item Be a faithful (as faithful as can be) representation of the system described. This means that a streamlined view of the system is attempted to be given here, and, in case of a discrepancy or lack of clarity, pointers on where to look next are given.
    \item Serve as a planning tool. The features listed here were most probably listed here before they made it into the actual codebase.
\end{enumerate}

In order to achieve this an unambiguous and simple language is sought, and the presence of additional standardized diagrams is also a must.
As for intentions that are out of the scope of this specification:
\begin{enumerate}
    \item Strictly avoiding vagueness: Some vaguenuess will creep in, and additional information (diagrams, parallels with the code) is given to hopefully alleviate the problem faced.
    \item Serve as a final description of the project: Considering the constrained situation, this specification is envisioned to be open-ended.
\end{enumerate}