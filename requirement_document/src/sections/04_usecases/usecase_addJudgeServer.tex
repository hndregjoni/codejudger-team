\begin{table}[htbp]
\centering
\begin{tabularx}{\textwidth}{|l|X|}
\hline
UC Name & UC\_Add\_judge\_server \\ \hline

Summary &  Judge servers are aimed to be horizontally scalable, even though one can do the job. An Admin can add additional ones. \\ \hline

Dependency & \- Health metrics usecase. \\ \hline

Actors & Admin \\ \hline

Preconditions & \- The server to be added must be set up. \newline \- No two servers can share the same (host, port) tuple \\ \hline

Description of the Main Sequence & 1. Admin opens list of Judge servers. \newline 2. Admin clicks the "Add" button. \newline 3. Admin is asked for the hostname and the port. \newline 4. Admin is asked for an authentication mechanism (no auth, basic authentication, cryptographic key, or auth token can be implemented.) \newline 5. Admin can add priority configuration, and capacity. \newline 6. Admin can ping the server for availability. \newline 7. Admin adds languages supported by the server. \\ \hline

Description of the Alternative Sequence & The languages supported can be added: \newline \- Manually from a list. \newline \- Queried from the server. \\ \hline

Non functional requirements & \- Transport security and isolation of the judge servers. \newline \\ \hline

Postcondition & A functional judge server is added, and ready to receive queued problems. \\ \hline

\end{tabularx}
\end{table}

