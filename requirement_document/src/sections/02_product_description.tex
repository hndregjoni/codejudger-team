\section{Product/Service Description}
In this section, describe the general factors that affect the product and its requirements. This section should contain background information, not state specific requirements (provide the reasons why certain specific requirements are later specified).

\subsection{Product Context}
How does this product relate to other products? Is it independent and self-contained?  Does it interface with a variety of related systems?  Describe these relationships or use a diagram to show the major components of the larger system, interconnections, and external interfaces.

\subsection{User Characteristics}
Create general customer profiles for each type of user who will be using the product. Profiles should include:

\begin{enumerate}
    \item{Student/faculty/staff/other}
    \item{experience}
    \item{technical expertise}
    \item{other general characteristics that may influence the product}
\end{enumerate}

\subsection{Assumptions}
List any assumptions that affect the requirements, for example, equipment availability, user expertise, etc.  For example, a specific operating system is assumed to be available; if  the operating system is not available, the Requirements Specification would then have to change accordingly.

\subsection{Constraints}
Describe any items that will constrain the design options, including

\subsection{Dependencies}
List dependencies that affect the requirements.  Examples: