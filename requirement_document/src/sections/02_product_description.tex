\section{Product/Service Description}

This project is primarily organized as a web service. The interface is to be served via the browser (API endpoints is available to facilitate further mediums). As hinted, one primary aspect of the system will be the 
backend server, serving the API.

The architecture of the system is to be monolithic, with some specific cases where a granular approach is chosen to enable for a more appropriate seperation of the domains. This is exemplified in the
fact that there will be a database server, a caching server, and also the novel judging servers. The latter are the more pertinent example. Their existence assures a more horizontally scalable approach to answer validation and code
running, as well as a more isolated environment for the execution of the code, which greatly limits potential attack surfaces for the system.

The components are as follows:
\begin{enumerate}
    \item The Gateway
    \item Backend server
    \item Frontend server
    \item Database server
    \item Cache server
    \item Judge servers
\end{enumerate}

\subsection{Product Context}

\subsection{User Characteristics}
Create general customer profiles for each type of user who will be using the product. Profiles should include:

\begin{enumerate}
    \item{Student/faculty/staff/other}
    \item{experience}
    \item{technical expertise}
    \item{other general characteristics that may influence the product}
\end{enumerate}

\subsection{Assumptions}
List any assumptions that affect the requirements, for example, equipment availability, user expertise, etc.  For example, a specific operating system is assumed to be available; if  the operating system is not available, the Requirements Specification would then have to change accordingly.

\subsection{Constraints}
Describe any items that will constrain the design options, including

\subsection{Dependencies}
List dependencies that affect the requirements.  Examples: